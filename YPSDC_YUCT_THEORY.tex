\documentclass[12pt, a4paper]{article}
\usepackage[utf8]{inputenc}
\usepackage{amsmath, amssymb, amsthm, bm}
\usepackage{hyperref}
\usepackage{graphicx}

% Theorem environments
\newtheorem{theorem}{Theorem}
\newtheorem{definition}{Definition}

% Equation numbering by section
\numberwithin{equation}{section}

\title{Geometric and Information-Theoretic Foundations of a Coordination-First Theory of Reality}

\author{Alexey V. Yakushev\thanks{Independent Researcher. Email: \texttt{alexey@yakushev.eu}. Websites: \url{https://ypsdc.com/}, \url{http://yuct.org/}.}}

\date{January 24, 2026}

\begin{document}
	
	\maketitle
	
	\begin{abstract}
		This work introduces a novel mathematical framework positing that coordination---the process of synchronizing states across distributed components---is ontologically prior to both spacetime and matter. This work proposes a fundamental ontological shift where coordination replaces spacetime and matter as primitive constructs. We formalize this through three interlocked constructs: (1) the \textbf{Dictionary Manifold} $\mathcal{M}_{\mathcal{D}}$, a Riemannian manifold where points represent dictionaries mapping compact indices to complex actions, equipped with a metric $g^{\mathcal{D}}_{ij}$ encoding semantic distance; (2) the \textbf{YPSDC Protocol} (Yakushev Protocol for Synchronous Distributed Coordination), which separates coordination from data transmission, enabling a coordination efficiency factor $K_{\mathrm{eff}} = H(\mathcal{A})/H(\kappa) > 1$ without violating causality; and (3) the \textbf{D+I•R Triad} $\text{Reality} = D + I \times R$, an ontological primitive combining Dictionary, Information, and Resonance multiplicatively. The complete structure is a fiber bundle $\pi:\mathcal{E}\to\mathcal{M}$ with spacetime as the base and $\mathcal{M}_{\mathcal{D}}$ as the fiber. This axiomatic core provides a unified geometric and information-theoretic language for subsequent derivation of physics, where laws and constants emerge from coordination principles. The framework is inherently testable through $K_{\mathrm{eff}}$-dependent corrections to gravitational and quantum equations.
	\end{abstract}
	
	\noindent\textbf{Keywords:} Coordination Theory, Dictionary Manifold, YPSDC Protocol, D+I•R Triad, Fiber Bundle, Information Geometry, Ontological Primitive, YUCT, $K_{\mathrm{eff}}$.
	
	\section{Introduction: The Primacy of Coordination}
	
	We propose a paradigm shift: the coordination of states among system components is ontologically prior to both spacetime and matter. To formalize this, we introduce a geometric and information-theoretic framework based on three interlocking constructs: the Dictionary Manifold, the YPSDC protocol, and the D+I•R triad.
	
	\section{The Dictionary Manifold $\mathcal{M}_{\mathcal{D}}$}
	
	\begin{definition}[Dictionary Manifold]
		A dictionary manifold $\mathcal{M}_{\mathcal{D}}$ is a smooth, finite-dimensional Riemannian manifold where each point $p \in \mathcal{M}_{\mathcal{D}}$ represents a complete \textit{dictionary}---a bijective mapping $\mathcal{D}: \mathcal{K} \to \mathcal{A}$ from a set of compact indices $\mathcal{K}$ to a set of complex actions or states $\mathcal{A}$.
	\end{definition}
	
	The manifold is equipped with a metric $g^{\mathcal{D}}_{ij}$ derived from the Fisher information of the dictionary's action space, measuring semantic distance between dictionaries. The curvature of $\mathcal{M}_{\mathcal{D}}$ encodes the complexity and relational structure of coordination protocols.
	
	\section{The YPSDC Protocol and $K_{\mathrm{eff}} > 1$}
	
	\begin{definition}[YPSDC Protocol]
		The Yakushev Protocol for Synchronous Distributed Coordination consists of:
		\begin{enumerate}
			\item \textbf{Offline Phase:} Distribution of a shared dictionary $\mathcal{D}$.
			\item \textbf{Online Phase:} Transmission of only a short index $\kappa \in \mathcal{K}$.
		\end{enumerate}
	\end{definition}
	
	\begin{definition}[Coordination Efficiency]
		For a system with separation $L$, channel capacity $C$, and processing delay $\tau$, the operational coordination efficiency is:
		\begin{equation}
			K_{\mathrm{eff}} = \frac{T_{\text{base}}}{T_{\text{actual}}} = \frac{H(\mathcal{A})/C + \tau}{H(\kappa)/C + \tau},
			\label{eq:keff}
		\end{equation}
		where $H(\mathcal{A})$ and $H(\kappa)$ are Shannon entropies.
	\end{definition}
	
	\begin{theorem}[Achievability of $K_{\mathrm{eff}} > 1$]
		For any system with a pre-shared dictionary $\mathcal{D}$ where $H(\mathcal{A}) \gg H(\kappa)$, the YPSDC protocol achieves $K_{\mathrm{eff}} \gg 1$ while maintaining causal signal propagation $v \leq c$.
	\end{theorem}
	
	\section{The D+I•R Triad}
	
	Physical reality is postulated to be described by the multiplicative triad:
	\begin{equation}
		\boxed{\mathrm{Reality} = D + I \times R}
		\label{eq:triad}
	\end{equation}
	where:
	\begin{itemize}
		\item $D$ (Dictionary): The structured set of potentialities (laws, codes, symmetries), modeled by $\mathcal{M}_{\mathcal{D}}$.
		\item $I$ (Information): The Shannon/von Neumann entropy content.
		\item $R$ (Resonance): A nonlinear amplification operator $R \geq 1$, representing coherence and synchronization gain.
	\end{itemize}
	
	The multiplicative coupling $I \times R$ enables effective information capacity exceeding naive channel limits, mathematically encapsulating the $K_{\mathrm{eff}}>1$ effect.
	
	\section{Unified Geometric Structure}
	
	The complete framework is a fiber bundle $\pi: \mathcal{E} \to \mathcal{M}$:
	\begin{itemize}
		\item \textbf{Base $\mathcal{M}$:} A 4-dimensional Lorentzian manifold (emergent spacetime).
		\item \textbf{Fiber $\mathcal{M}_{\mathcal{D}}(x)$:} Dictionary manifold attached at each $x \in \mathcal{M}$.
		\item \textbf{Total Space $\mathcal{E}$:} All dictionary-field configurations over spacetime.
		\item \textbf{Section $\sigma(x)$:} A choice of dictionary $\mathcal{D}(x)$ at each point.
	\end{itemize}
	
	The geometry of $\mathcal{E}$ combines causal structure of $\mathcal{M}$ with semantic-informational structure of the fibers.
	
	\section{Conclusion and Path Forward}
	
	This paper establishes the axiomatic mathematical core of a coordination-first ontology. The constructs of Dictionary Manifold, YPSDC protocol with $K_{\mathrm{eff}}$, and D+I•R triad provide a unified language for subsequent development. Immediate applications to be developed in specialized publications include:
	
	\begin{enumerate}
		\item Derivation of Einstein field equations with $K_{\mathrm{eff}}$-corrections from variational principles on $\mathcal{E}$
		\item Recasting of quantum mechanics as the $R \to \infty$ limit of D+I•R dynamics
		\item Explicit testable predictions for perihelion precession: $\Delta\phi_{\mathrm{total}} = \Delta\phi_{\mathrm{GR}}(1 + \frac{4}{3}\kappa^2)$
		\item Modeling of biological and social systems as high-$K_{\mathrm{eff}}$ regimes
	\end{enumerate}
	
	The theory is falsifiable through $K_{\mathrm{eff}}$-dependent corrections to established physical laws, with the strongest constraints expected from precision orbital mechanics.
	
	\section*{Acknowledgments}
	The author thanks the anonymous reviewers for their constructive feedback.
	
	\begin{thebibliography}{20}
		
		\bibitem{Amari2016}
		Amari, S. (2016).
		\newblock \emph{Information Geometry and Its Applications}.
		\newblock Springer, Berlin.
		
		\bibitem{Kobayashi1963}
		Kobayashi, S., and Nomizu, K. (1963).
		\newblock \emph{Foundations of Differential Geometry}, Vol. 1.
		\newblock Wiley, New York.
		
		\bibitem{Jacobson1995}
		Jacobson, T. (1995).
		\newblock Thermodynamics of spacetime: The Einstein equation of state.
		\newblock \emph{Physical Review Letters}, 75(7), 1260--1263.
		
		\bibitem{Einstein1915}
		Einstein, A. (1915).
		\newblock Die Feldgleichungen der Gravitation.
		\newblock \emph{Sitzungsberichte der K\"oniglich Preussischen Akademie der Wissenschaften}, 844--847.
		
		\bibitem{Shannon1948}
		Shannon, C. E. (1948).
		\newblock A mathematical theory of communication.
		\newblock \emph{Bell System Technical Journal}, 27(3), 379--423.
		
		\bibitem{Misner1973}
		Misner, C. W., Thorne, K. S., and Wheeler, J. A. (1973).
		\newblock \emph{Gravitation}.
		\newblock W. H. Freeman, San Francisco.
		
		\bibitem{Wald1984}
		Wald, R. M. (1984).
		\newblock \emph{General Relativity}.
		\newblock University of Chicago Press, Chicago.
		
		\bibitem{vonNeumann1932}
		von Neumann, J. (1932).
		\newblock \emph{Mathematische Grundlagen der Quantenmechanik}.
		\newblock Springer, Berlin.
		
		\bibitem{Nash1956}
		Nash, J. (1956).
		\newblock The imbedding problem for Riemannian manifolds.
		\newblock \emph{Annals of Mathematics}, 63(1), 20--63.
		
		\bibitem{Weinberg1972}
		Weinberg, S. (1972).
		\newblock \emph{Gravitation and Cosmology: Principles and Applications of the General Theory of Relativity}.
		\newblock Wiley, New York.
		
	\end{thebibliography}
	
\end{document}